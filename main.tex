\documentclass[a4paper,11pt]{article}
\usepackage[T1]{fontenc}
\usepackage[utf8]{inputenc}
\usepackage{lmodern}
\usepackage{amsmath}
\usepackage{amsfonts}
\usepackage{amssymb}
\usepackage{amsthm}
\usepackage{graphicx}
\usepackage{color}
\usepackage{xcolor}
\usepackage{url}
\usepackage{theorem}
\usepackage{textcomp}
\usepackage{listings}
\usepackage{hyperref}
\usepackage{glossaries}
\usepackage{parskip}
\usepackage{tikzit}
\input{tikz/style.tikzstyles}

\title{Cosc341 - Assignment 3}
\author{Dan Skaf, Elijah Passmore, Tom Berg}
\date{\today}

\begin{document}

\maketitle

\begin{abstract}
This report introduces the \textbf{Longest Path problem} and proves it is $NP$-complete.
\end{abstract}

\section{Longest Path problem}
The \textbf{Longest path problem} is a problem for finding the maximum length 
Simple-path in a graph, $G(V,E)$ with $\left|V\right|=n\::\:n\:\in\:\mathbb{N}$.
The graph is undirected and all edges weights are assumed to be $1$.

    \subsection{Simple-path} 
    Suppose in our graph, $G(V,E)$, we want the Simple-path between two vertices 
    $s$ and $t$. The Simplest-path is a path between any two nodes in the graph 
    that does not go over the same node twice, and for two consecutive nodes 
    there is a path between them.

    As such a simple path between two nodes $s,t\in V$ is the sequence of 
    vertices ($v_1$,$v_2$,$v_3$,$....$,$v_k$) that satifisy these conditions:
    \begin{itemize}
        \item $s\:=v_1$ and $t\:=\:v_k$
        \item Each consecutive nodes $\left(v_i,v_{i+1}\right)$ there is an 
        edge $e\:=\left(v_i,v_{i+1}\right)\:\in E$
        \item No node appears more than once in the sequence.
    \end{itemize}
    \pagebreak

    \subsection{Longest-simple path and Decision version}
    The figure below shows the longest simple path, in blue, of a graph.
    \begin{figure}[!h]
        \centering \tikzfig{tikz/longestPathIntro}
        \caption{Longest simple path of a graph}
    \end{figure}

    The decision version of the Longest path problem, yes or no version; 
    input a graph, $G$, and integer $k$ outputs a yes or no if a simplest path 
    of at least length, $k$, exists in the graph.
    \subsection{NP}
    A problem is in NP if we can verify the solution in polynomial time. A 
    corresponding certificate can be produced from the decision version of the 
    Longest path problem if we can verify this certificate in polynomial time 
    then Longest path problem is in $NP$. The certificate is a path at least of 
    length $k$ with no duplicate nodes.\\ 
    To check that the path has no duplicate 
    nodes we can itterate over each node in the path for every node, $O(k^2)$ 
    complexity. 
    \\To check the length of the path we just need to count each node
    and check it is greater than or equal to $k$ with a $O(k)$ complexity. 
    \\In some impletations like most programming lanuages getting the length of 
    a list is constant complexity $O(1)$, in most cases and the worst 
    case we can check a certificate with $O(n^2)$ complexity, therefore the
    Longest path problem is in $NP$.

\end{document}
