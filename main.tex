% ---------------------------------------------------------------------------- %
%                                   PREAMBLE                                   %
% ---------------------------------------------------------------------------- %
\documentclass[a4paper,11pt]{article}
\usepackage[T1]{fontenc}
\usepackage[utf8]{inputenc}
\usepackage{lmodern}
\usepackage{amsmath}
\usepackage{amsfonts}
\usepackage{amssymb}
\usepackage{amsthm}
\usepackage{graphicx}
\usepackage{color}
\usepackage{xcolor}
\usepackage{url}
\usepackage{theorem}
\usepackage{textcomp}
\usepackage{listings}
\usepackage{bookmark}
\usepackage{glossaries}
\usepackage{parskip}
\usepackage{tikzit}
\input{tikz/style.tikzstyles}

% ---------------------------------------------------------------------------- %
%                                    CONTENT                                   %
% ---------------------------------------------------------------------------- %
\title{COSC341 - Assignment 3}
\author{Tom Berg, Elijah J. Passmore, Dan Skaf}
\date{\today}

\begin{document}

\maketitle
\tableofcontents

% --------------------------- Longest Path Problem --------------------------- %
\section{Longest Path Problem}

The \textbf{Longest Path Problem} is for finding the maximum length simple path
in a graph, $G(V,E)$ with $\left|V\right|=n\::\:n\:\in\:\mathbb{N}$.  The graph
is undirected and all edges weights are assumed to be $1$.

\subsection{Simple Path}

Suppose we want the \textbf{simple path} between two vertices $s$ and $t$.  A
simple path is one which every node in the path is never visited more than once
and that there is a path between two consecutive nodes.

Therefore, a simple path between two nodes $s,t\in V$ is the sequence of
vertices ($v_1$,$v_2$,$v_3$,$....$,$v_k$) that satisfies these conditions:
\pagebreak

\begin{itemize}
    \item $s\:=v_1$ and $t\:=\:v_k$
    \item Each consecutive nodes $\left(v_i,v_{i+1}\right)$ there is an edge
          $e\:=\left(v_i,v_{i+1}\right)\:\in E$
    \item No node appears more than once in the sequence.
\end{itemize}

\subsection{Longest Simple Path and Decision Version}

The figure below shows the longest simple path in blue, beginning in the upper
left:

\begin{figure}[!h]
    \centering \tikzfig{tikz/longestPathIntro}
    \caption{Longest simple path}
\end{figure}

In the decision version of the Longest Path Problem, we input a graph $G$, and
an integer $k$, and output a \textbf{yes or no} if a simplest path of
\textbf{at least} length $k$ exists in the graph.

\subsection{NP}

A problem is in \verb|NP| if we can verify the solution in polynomial time, and
if a corresponding certificate can be produced from the decision version of the
Longest Path Problem. If we can verify this certificate in polynomial time, then
the Longest Path Problem is in \verb|NP|. The certificate is a path of at least
length $k$ with no duplicate nodes.

To check that the path has no duplicate nodes we can iterate over each node in
the path for every node, $O(k^2)$ complexity. To check the length of the path we
just need to count each node and check it is greater than or equal to $k$, this
operation has a $O(k)$ complexity.

In some implementations, getting the length of a list is constant complexity
$O(1)$. In most cases and the worst case we can check a certificate with is
$O(n^2)$ complexity, therefore the Longest Path Problem is in \verb|NP|.

% -------------------------------- NP-Complete ------------------------------- %
\section{NP-Complete}

If a problem, $X$, is in \verb|NP-Complete| and $X$ polynomially reduces to another
problem $Y$ in \verb|NP|, then $Y$ is also in \verb|NP-Complete|.

We have shown that the Longest Path Problem is in \verb|NP|, all that is left is
to show that another problem in \verb|NP-Complete| polynomial reduces to the
Longest Path Problem.

\subsection{Hamiltonian Path}

A \textbf{Hamiltonian path} is a sequence that visits each node \textbf{exactly}
once. The \textbf{Hamiltonian Path Problem} is a known \verb|NP-Complete| problem that
asks if a Hamiltonian path exists for a given graph.

\subsection{Reduction to Longest Path Problem}

With any instance of the Hamiltonian Path Problem of an undirected graph of size
$n$, we can instead ask if a longest path of size $k\:=\:n-1$ exists.

When we set $k$ to $n-1$, we are asserting that the longest path is equal to the
Hamiltonian path.

\subsection{Correctness}

To prove our reduction we must show that:

\begin{center}

    Correct decision version of Hamiltonian path

    $\updownarrow$

    Correct decision version of Longest Path

\end{center}

That is, the instance of the Hamiltonian Path Problem exits only if a longest
path of $k\:=\:n-1$ exists. We will prove this in both directions from the
definitions.

\subsubsection{A Longest Path Exists}

\begin{itemize}
    \item If a longest path of size $k\;=n-1$ exists, there is a simple path
          that passes each node exactly once.
    \item This simple path is a Hamiltonian path, and if a longest path does not
          exist there cannot be a Hamiltonian path.
\end{itemize}
\pagebreak

\subsubsection{A Hamiltonian Path Exists}

\begin{itemize}
    \item If a Hamiltonian path exists, there is a simple path of length $n-1$.
    \item This is the maximum possible length of any simple path in the graph,
          therefore it is a longest path of size $k\:=\:n-1$.
    \item If a Hamiltonian path does not exist, it is impossible for a longest
          path of $n-1$ to exist, illustrated by a disconnected graph in $fig.2$
\end{itemize}

\begin{figure}[!h]
    \centering \tikzfig{tikz/DisconnectedGraph}
    \caption{Graph without a Hamiltonian path, longest path on the right.}
\end{figure}

Therefore we have proved that our reduction is correct.

% -------------------------------- Conclusion -------------------------------- %
\section{Conclusion}

We have shown that the decision version of any Longest Path Problem has a
certificate that is checked in polynomial time, therefore it is in \verb|NP|. We
showed a reduction from the Hamiltonian Path Problem to the Longest Path
Problem, and proved the correctness of the reduction. Therefore, we have proved
that the Longest Path Problem is also \verb|NP-Complete| $\blacksquare$

\end{document}
