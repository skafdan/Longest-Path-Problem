\documentclass[a4paper,11pt]{article}
\usepackage[T1]{fontenc}
\usepackage[utf8]{inputenc}
\usepackage{lmodern}
\usepackage{amsmath}
\usepackage{amsfonts}
\usepackage{amssymb}
\usepackage{amsthm}
\usepackage{graphicx}
\usepackage{color}
\usepackage{xcolor}
\usepackage{url}
\usepackage{theorem}
\usepackage{textcomp}
\usepackage{listings}
\usepackage{hyperref}
\usepackage{glossaries}
\usepackage{parskip}
\usepackage{tikzit}
\input{tikz/style.tikzstyles}

\title{Cosc341 - Assignment 3}
\author{Dan Skaf, Elijah Passmore, Tom Berg}
\date{\today}

\begin{document}

\maketitle

\section{Longest Path problem}
The \textbf{Longest path problem} is the problem for finding the maximum length 
Simple-path in a graph, $G(V,E)$ with $\left|V\right|=n\::\:n\:\in\:\mathbb{N}$.
The graph is undirected and all edges weights are assumed to be $1$.

    \subsection{Simple-path} 
    Suppose in our graph, $G(V,E)$, we want the Simple-path between two vertices 
    $s$ and $t$. The Simple-path is a path between any two nodes in the graph 
    that does not go over the same node twice, and for two consecutive nodes 
    there is a path between them.

    As such a simple path between two nodes $s,t\in V$ is the sequence of 
    vertices ($v_1$,$v_2$,$v_3$,$....$,$v_k$) that satisfy these conditions:
    \begin{itemize}
        \item $s\:=v_1$ and $t\:=\:v_k$
        \item Each consecutive nodes $\left(v_i,v_{i+1}\right)$ there is an 
        edge $e\:=\left(v_i,v_{i+1}\right)\:\in E$
        \item No node appears more than once in the sequence.
    \end{itemize}

    \subsection{Longest-simple path and Decision version}
    The figure below shows the longest simple path, in blue, of a graph.
    \begin{figure}[!h]
        \centering \tikzfig{tikz/longestPathIntro}
        \caption{Longest simple path of a graph}
    \end{figure}

    In the decision version of the Longest path problem, yes or no version; 
    we input a graph, $G$, and an integer $k$ and output a yes or no if a 
    simplest path of \textbf{at least} length, $k$, exists in the graph.
    \subsection{NP}
    A problem is in NP if we can verify the solution in polynomial time. A 
    corresponding certificate can be produced from the decision version of the 
    Longest path problem. If we can verify this certificate in polynomial time 
    then Longest path problem is in $NP$. The certificate is a path at least of 
    length $k$ with no duplicate nodes.\\ 
    To check that the path has no duplicate nodes we can iterate over each node
     in the path for every node, $O(k^2)$ complexity. 
    \\To check the length of the path we just need to count each node
    and check it is greater than or equal to $k$, this operation has a $O(k)$ 
    complexity. 
    \\In some implementations like most programming languages getting the length of 
    a list is constant complexity $O(1)$, in most cases and the worst 
    case we can check a certificate with $O(n^2)$ complexity, therefore the
    Longest path problem is in $NP$.

\section{NP-complete}
If a problem, $X$, is in $NP$-complete and X polynomial reduces to another 
problem $Y$ in NP, then Y is also in $NP$-complete.\\
We have shown that the Longest path problem is $NP$, all thats left is to show 
that another problem in $NP$-complete polynomial reduces to the Longest path 
problem.
\subsection{Hamiltonian path}
A Hamiltonian path problem - Determining if a Hamiltonian path:\\ 
\textbf{A path which visits every node in a graph exactly once}\\ exists in a 
graph, is a known $NP$-complete problem. It stands to show that the Hamiltonian 
path problem polynomial reduces to the Longest path problem.
\subsection{Reduction to Longest Path Problem}
With any instance of the Hamiltonian path problem, an undirected graph of size 
$n$, we can instead ask if a Longest path, of size $k\:=\:n-1$, exists.\\
When we set $k$ to $n-1$ we are asserting that the Longest path
problem is equal to a Hamiltonian path.
\pagebreak
\subsection{Correctness}
To prove our reduction we must show:\\
\textbf{Correct decision version of Hamiltonian path $\Longleftrightarrow$ 
\\Correct decision version of Longest Path}\\
That is a instance of the Hamiltonian path problem exits if and only if a 
Longest path, of $k\:=\:n-1$, exists. We will prove this in both directions 
from the definitions. 
\subsubsection{A Longest path exists}
\begin{itemize}
    \item If a Longest Path of size $k\;=n-1$ exists, there is a simple path 
    that passes each node exactly once.
    \item This simple path is a Hamiltonian path, as such if a Longest path does
     not exist there cannot be a Hamiltonian path.
\end{itemize}
\subsubsection{A Hamiltonian path exists}
\begin{itemize}
    \item If a Hamiltonian path exists, there is a simple path of length $n-1$
    \item This is the maximum possible length of any simple path in the graph,
    as such it is a Longest Path of size $k\:=\:n-1$.
    \item If a Hamiltonian path does not exist it is impossible for a Longest 
    path of $n-1$ to exist. Illustrated by a disconnected graph in $fig.2$
\end{itemize}
\begin{figure}[!h]
        \centering \tikzfig{tikz/DisconnectedGraph}
        \caption{Graph without a Hamiltonian path, Longest path on right.}
    \end{figure}
Therefore we have proved our reduction is correct.
\section{Conclusion}
We have shown that the decision version of any Longest path problem has a 
certificate that is checked in polynomial time, therefore it is in $NP$. We also
showed a reduction from a well known $NP-complete$ problem; Hamiltonian path
problem, to the Longest path problem and proved the correctness of the reduction.
As such we have proved that the Longest path problem is also $NP-complete$ 
$\blacksquare$
\end{document}
